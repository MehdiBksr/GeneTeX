\documentclass[12pt]{article}

\usepackage[french]{babel}
\usepackage[T1]{fontenc}
\usepackage[utf8]{inputenc}
\usepackage{fancyhdr}
\usepackage{lastpage}
\usepackage{graphicx}
\usepackage{verbatim}
\usepackage{amsmath}

\usepackage{listings}
\lstset{
  sensitive=f,
  morecomment=[l]--,
  morestring=[d]",
  showstringspaces=false,
  basicstyle=\small\ttfamily,
  keywordstyle=\bf\small,
  commentstyle=\itshape,
  stringstyle=\sf,
  extendedchars=true,
  columns=[c]fixed
}


% CI-DESSOUS: conversion des caractères accentués UTF-8 
% en caractères TeX dans les listings...
\lstset{
  literate=%
  {À}{{\`A}}1 {Â}{{\^A}}1 {Ç}{{\c{C}}}1%
  {à}{{\`a}}1 {â}{{\^a}}1 {ç}{{\c{c}}}1%
  {É}{{\'E}}1 {È}{{\`E}}1 {Ê}{{\^E}}1 {Ë}{{\"E}}1% 
  {é}{{\'e}}1 {è}{{\`e}}1 {ê}{{\^e}}1 {ë}{{\"e}}1%
  {Ï}{{\"I}}1 {Î}{{\^I}}1 {Ô}{{\^O}}1%
  {ï}{{\"i}}1 {î}{{\^i}}1 {ô}{{\^o}}1%
  {Ù}{{\`U}}1 {Û}{{\^U}}1 {Ü}{{\"U}}1%
  {ù}{{\`u}}1 {û}{{\^u}}1 {ü}{{\"u}}1%
}

%%%%%%%%%%
% TAILLE DES PAGES (A4 serré)

\setlength{\parindent}{0cm}
\setlength{\parskip}{1ex}
\setlength{\textwidth}{17cm}
\setlength{\textheight}{24cm}
\setlength{\oddsidemargin}{-.7cm}
\setlength{\evensidemargin}{-.7cm}
\setlength{\topmargin}{-.5in}

%%%%%%%%%%
% EN-TÊTES ET PIED DE PAGES

\pagestyle{fancyplain}
\renewcommand{\headrulewidth}{0pt}
\addtolength{\headheight}{1.6pt}
\addtolength{\headheight}{2.6pt}
\lfoot{}
\cfoot{}
\rfoot{\footnotesize\sf \thepage/\pageref{LastPage}} % numero de page
\lhead{\footnotesize\sf Projet Génie Logiciel} % en-tete Gauche
\rhead{\footnotesize\sf Équipe GL36} % en-tete Droit

%%%%%%%%%%
% TITRE DU DOCUMENT

\title{\textbf{Decac\\
  Manuel Utilisateur}}

\author{Mehdi \textsc{Bouksara} - Étienne \textsc{Delaporte} - Hugues \textsc{De Lassus Saint-Geniès} \\
  Théo \textsc{Merle} - Morgane \textsc{Sanfins}}

\date{Janvier 2013}


%%%%%%%

\begin{document}

\maketitle
~\\
\begin{center}
\includegraphics[width=10cm]{coffee.eps}
\end{center}

\newpage
\section{Introduction}

Decac est un compilateur pour le langage Deca (langage inspiré de Java).\\
Pour compiler un fichier .deca, il faut utiliser la syntaxe suivante :\\
\lstinline{$decac <options> <fichier_source.deca>}\\%$
où les options possibles sont décrites dans la section \textit{Options} de ce manuel.\\

Si le programme source ne contient pas d'erreurs,\\
\lstinline{$decac [repertoire/]fichier_source.deca}\\ %$
ne produit aucune sortie et crée un fichier fichier\_source.ass dans le
répertoire depuis lequel la commande est lancée, interprétable par une machine
abstraite.\\
Sinon, decac affiche sur la sortie standard les erreurs à corriger présentées 
dans la section \textit{Messages d'erreurs} de ce manuel.


% \subsection{Package \lstinline!Code!}

\section{Options}

\subsection{Synopsis}
\lstinline{$decac [[-p | -v] [-n] [-r X] [-x] [-w] [repertoire/]nom_fichier_source.deca] | [-b]}\\ %$

\subsection{Tags}
\begin{itemize}
       \item[] \lstinline!-b! (banner)\\
         Affiche la bannière de Decac indiquant notre nom d'équipe. 
         S'utilise sans aucun autre argument.
       \item[] \lstinline!-p! (parse)\\
         Arrête decac après l'étape de construction de
         l'arbre, et affiche la décompilation de ce dernier.
         Cette option n'est pas compatible avec l'option \lstinline!-v!.
       \item[]\lstinline!-v! (verification)\\
         decac s'arrête après l'étape de vérifications contextuelles 
         (ne produit aucune sortie en l'absence d'erreur).
         Cette option n'est pas compatible avec l'option \lstinline!-p!.
       \item[]\lstinline!-n! (no check)\\
         Supprime les tests de débordement à l'exécution : 
         \begin{itemize}
           \item débordement arithmétique
           \item débordement mémoire
           \item déréférencement de null
         \end{itemize}
       \item[]\lstinline!-r X! (registers)\\
         Limite les registres banalisés disponibles à R0, ..., R\{X-1\}, avec
         $4 \leq X \leq 16$.
       \item[]\lstinline!-x (heXadecimal)!\\
         Passe l'affichage des flottants en hexadécimal (décimal par défaut)
       \item[]\lstinline!-w! (warnings)\\
         Autorise l'affichage de messages d'avertissement en cours de compilation
\end{itemize}

\section{Types et fonctions prédéfinis}
\subsection{Types et classes}
Le compilateur Decac reconnaît par défaut les types suivants :\\
\\- void : type vide, utilisé pour déclarer des méthodes ne renvoyant aucune valeur.\\
\\- boolean : type booléen, utilisé dans les expressions boolean. Peut prendre deux valeurs (true ou
false).\\
\\- float : type flottant, ces nombres doivent comporter un . ($3.1$,$5.0$...),
ils peuvent comporter une puissance de 10 ($5.1E-5$, $3.2e21$...) et peuvent
être écrits au format hexadécimal($0x1.312Fa$)
avec une puissance de 2 ($0x1.312F1p+12$).\\
\\- int : type entier, ils ne peuvent pas commencer par le chiffre '0' (sauf le nombre 0 lui-même), ils ne doivent pas contenir de ., leurs bornes sont indiquées dans la section Limitations.\\
\\- Object : classe mère objet, toutes les classes que l'on déclarera dans un fichier .deca en hériteront. Cette classe est détaillée dans la section Fonctions et méthodes.\\
\\- Math : classe dédiée aux calculs mathématiques, elle est munie de quatre méthodes détaillées dans la section Fonctions et méthodes.\\
\\
Les chaînes de caractères peuvent être imprimées via la fonction prédéfinie print, mais il n'existe pas de type prédéfini permettant de stocker
des chaînes de caractères en mémoire.
\subsection{Fonctions et méthodes}
Les opérations classiques de comparaison arithmétiques sont accessibles (\verb?<?, \verb?>?, \verb?<=?, \verb?>=?) elles doivent comparer des flottants 
ou des entiers.Dans le cas où les deux opérandes sont de type différent, la conversion de l'entier en flottant se fait de manière implicite.\\
Les opérations \verb?==? et \verb?!=? acceptent en opérandes des booléens, des entiers, des flottants, ou des objets de classe quelconque. Cependant, on ne peut 
comparer que deux variables de même type prédéfini (sauf int et float) ou deux objets.
Attention cependant, s'il s'agit d'objets, ces opérations ne comparent que les adresses en mémoire. Pour comparer deux objets de façon différente, il est nécessaire de 
redéfinir la méthode equals de la classe Object.\\
\\
Les opérations classiques sur les booléens sont aussi accessibles (\verb?||?, \verb?&&?, \verb?!?). Les opérations booléennes s'effectuent avec les règles de priorité classiques.\\

Enfin, les opérations de différence (\verb?-?), d'addition (\verb?+?), de multiplication (\verb?*?) et de division (\verb?/?)
sont autorisées entre entiers ou flottants (avec encore une conversion implicite pour une opération mixte). Le modulo (\verb?%?) est autorisé entre entiers.\\

Les autres fonctions ou méthodes prédéfinies sont les suivantes :\\
\\- print : permet d'afficher sur la sortie standard des entiers, flottants ou chaînes de caractère. Cette fonction prend autant d'arguments qu'il le faut, séparés par des virgules.\\ print() est équivalent à l'instruction vide.
\\- println : même fonction que print, mais celle-ci saute une ligne à la fin de l'affichage.\\ println() est équivalent à un saut de ligne sur la sortie standard.
\\- readInt : cette fonction est interactive, elle attend que l'utilisateur entre un entier et le renvoie.\\ Le programme lèvera une erreur à l'exécution si un flottant ou une chaîne de caractères quelconques est entrée.
\\- readFloat : cette fonction est semblable à readInt mais attend un flottant. Attention, le programme lèvera une erreur à l'exécution si un entier ou une chaîne de caractères quelconque est entrée, ainsi que pour des flottants entrés en écriture hexadécimale.
\\- cast : le cast s'utilise ainsi : \verb?cast<type>(expression)? et renvoie \verb?expression? convertie pour avoir le type \verb?type?. La conversion n'est possible que si : \\
\begin{itemize}
     \item \verb?expression? est de type float et \verb?type? vaut int
     \item \verb?expression? est de type int et \verb?type? vaut float
     \item \verb?type? est un type de classe A quelconque et \verb?expression? est d'un type de classe héritée de \verb?type?
\end{itemize}

~\\
Attention, une conversion d'un objet vers une classe héritée ne provoquera pas d'erreur de compilation, mais le programme généré lèvera une erreur à l'exécution.
\\\\Il faut également porter une attention particulière sur le fait qu'une conversion \verb?b = cast<B>(a)? donnera accès aux champs et méthodes de la classe B, mais les champs et méthodes de la classe d'origine de l'objet a ne seront pas accessibles à B.
\\- instanceof : l'expression instance of s'utilise ainsi : \verb?a? \verb?instanceof? \verb?B? où a est un objet (ou l'expression null) et B un type de classe. L'expression vaut \verb?true? si la classe de a hérite de B, et \verb?false? sinon.
\\- equals : cette méthode de la classe Object est équivalent à l'opérateur \verb?==?, mais il est possible de la redéfinir lors de la déclaration d'une classe.\\
\\- sin : cette méthode de la classe Math renvoie le sinus du nombre donné en argument (entier ou flottant) en radians. Une précision d'au moins $7$ chiffres significatifs
sera obtenue entre -100 et 100 radians, au-delà, on peut s'attendre à une perte de précision.\\
\\- cos : cette méthode de la classe Math renvoie le cosinus du nombre donné en argument (entier ou flottant) en radians. La précision est semblable à celle de sin.\\
\\- asin : cette méthode de la classe Math renvoie l'arcsinus du nombre donné en argument (entier ou flottant), qui doit se trouver dans l'intervalle [-1, 1]. 
Les calculs sont précis à $7$ chiffres significatifs.\\
\\- atan : cette méthode de la classe Math renvoie l'arctangente du nombre donné en argument, une précision de $7$ chiffres significatifs sera obtenue sur tout 
l'intervalle de définition des flottants.\\

\subsection{Autres mots réservés du langage}
Le langage Deca comporte également d'autres mots réservés. Ces mots ne peuvent pas être utilisés pour déclarer une variable, un objet ou une classe : \\
\begin{itemize}
\item if : permet de commencer un branchement conditionnel.
\item elsif : permet de donner une alternative dans un branchement conditionnel.
\item else : permet de donner une alternative par défaut dans un branchement conditionnel.
\item extends : permet de déclarer une classe en spécifiant son extension.
\item true : désigne le booléen valant vrai.
\item false : désigne le booléen valant faux.
\item class : permet de déclarer une classe.
\item protected : permet de déclarer un champ de classe protégé, qui ne sera accessible que dans une classe héritant de la classe dans laquelle il est défini.
\item return : permet de spécifier une valeur de retour pour une méthode de classe de type différent de void.
\item this : expression représentant l'objet manipulé dans une classe.
\item while : permet de commencer une boucle.
\item null : représente l'absence d'adresse. Ne peut être affecté qu'à un objet. le type de null est considéré comme héritant de n'importe quel type de classe défini.
\item new : permet d'instancier un objet.
\end{itemize}
\section{Limitations et erreurs recensées}

\subsection{Limitations}
La taille maximale d'une chaîne de caractères est 1024 caractères, guillemets 
compris. Si une chaîne est trop grande, une erreur lexicale de débordement 
sera levée. Pour afficher une telle chaîne, il suffit de la séparer
en plusieurs arguments séparés par des virgules.

La compilation de plus d'une centaine d'expressions imbriquées est susceptible
de lever une erreur syntaxique de débordement.

Les nombres manipulés doivent être compris entre $-2147483648$ et $2147483647$, sinon,
une erreur lexicale de débordement sera levée.

Les flottants sont compris entre $-3.40282346 \times 10^{38}$ et $3.40282346
\times 10^{38}$, avec 7 chiffres significatifs.
Le plus petit nombre au dessus de zéro se situe autour de $1.4 \times 10^{-45}$.

La présence du caractère null dans un fichier lève une erreur lexicale, mais
le numéro de ligne de l'erreur n'apparaît pas.

Une opération de conversion d'un objet vers une classe héritée de la classe de cet objet sera acceptée par le compilateur,
mais une erreur à l'exécution sera levée.

Une opération arithmétique sur les entiers provoquant un dépassement de la capacité de représentation de la machine abstraite
ne provoquera pas d'erreur à l'exécution, mais les valeurs obtenues seront erronées.
\subsection{Erreurs recensées}

Aucune erreur interne au compilateur n'a été recensée durant le développement de celui-ci. L'existence de telles erreurs 
n'est cependant pas exclue.

\section{Messages d'erreurs retournés}

Cette section détaille les messages d'erreurs pouvant être retournés à 
l'utilisateur, ainsi que les configurations qui les provoquent.
\subsection{Erreurs d'usage du compilateur}
Les erreurs qui suivent proviennent d'une utilisation de la commande decac ne correspondant pas au format attendu.\\
\lstinline!decac: options non compatibles!\\
Provoquée par une utilisation de decac non conforme aux spécifications.\\
Par exemple : \lstinline!decac -p -v fichier.deca!\\

\lstinline!decac: fichier X.deca inexistant! ou\\
\lstinline!decac: extension.deca manquante! ou\\
\lstinline!decac: un seul fichier autorisé!\\
Le chemin vers le fichier à compiler doit être correct.\\
Decac ne compile que les fichiers .deca, un par un.

\lstinline!decac: valeur incorrecte pour l'option -r : X!\\
\lstinline!decac: mauvais usage de l'option -r!\\
L'option \lstinline!-r! attend un argument entier compris entre 4 et 16.\\

\lstinline!decac: flag invalide : -XXX!\\
Provoquée par l'utilisation d'une option inexistante. Les seules options 
possibles sont listées dans la section Options.

\subsection{Erreurs de lexicographie}
Ces erreurs proviennent du fichier donné en argument à Decac. Le nombre qui apparaît
correspond au numéro de la ligne qui provoque une erreur.

\lstinline!Ligne X : Erreur lexicale, débordement!\\
Un entier hors des limites
précisées dans Limitations ou une chaîne de plus de
1024 caractères a été repéré dans le fichier .deca.\\

\lstinline!Ligne ? : Erreur lexicale, le caractère null est présent quelque part dans le fichier.!\\
Le caractère null est présent dans le fichier
donné à compiler, il ne s'agit probablement pas d'un fichier Deca correct.\\

\lstinline!Ligne X : Erreur lexicale, caractère Y inattendu ou manquant!\\
On trouve un caractère invalide (caractère
non imprimable, caractères non acceptés dans les identifiants...)
ou lorsqu'une chaîne est mal terminée.\\

\subsection{Erreurs de syntaxe}
Le message suivant apparaît lorsqu'il existe un problème de syntaxe à la ligne X,
le fichier .decac est probablement mal écrit, il n'est pas correct pour ce langage.\\

\lstinline!Ligne X : Erreur syntaxique, problème de syntaxe aux alentours de 'Y'!\\
Il existe un problème de syntaxe dans le fichier
(parenthèse manquante, instruction ou déclaration inattendue à cet endroit...)
la chaîne Y permet de déterminer plus précisément l'endroit de l'erreur dans
le fichier, cependant s'il s'agit d'un caractère manquant et non en trop, la
chaine peut être vide.\\

\subsection{Erreurs de vérification contextuelle}
Les messages d'erreur suivants correspondent à des erreurs contextuelles, c'est à dire que
les actions du programme ne sont pas autorisées bien qu'il soit correctement écrit. X indique
le numéro de la ligne où l'erreur apparaît.\\
\\
\lstinline!Ligne X : Impossible d'hériter d'une super classe non déclarée auparavant.!\\
Une classe tente d'hériter d'une classe non encore déclarée.
L'héritage ne peut se faire que si la superclasse est déclarée avant la classe héritée.
 La suite du message indique le nom des classes concernées.\\

\lstinline!Ligne X : Impossible de déclarer une classe deux fois.!\\
La classe déclarée à la ligne X l'a déjà été, il est impossible
de redéfinir une classe. La suite du message indique la classe concernée.\\

\lstinline!Ligne X : Impossible de faire de la surcharge de méthode.!\\
Ce cas de figure correspond à une tentative de redéfinition d'une méthode d'une super classe, mais
avec des paramètres différents. La surcharge de méthode n'est pas autorisée dans le langage Deca.
La suite du message indique le nom de la classe concernée.\\

\lstinline!Ligne X : La méthode héritée n'a pas le bon type de retour.!\\
Lors de la redéfinition d'une méthode de la superclasse, le type
de retour de la nouvelle méthode n'est pas un sous-type de celui de la méthode héritée. La suite
du message indique de quelle méthode il s'agit.\\

\lstinline!Ligne X : Le champ Y existe déjà !
\lstinline!dans une super classe et n'est pas un champ.!\\
Un champ Y est déclaré dans une classe alors qu'une méthode nommée Y existe
déjà dans une de ses super-classes.\\

\lstinline!Ligne X : Cette méthode existe déjà.!\\
La méthode définie existe déjà dans cette classe.
La suite du message indique le nom de cette méthode.\\

\lstinline!Ligne X : Ce champ existe déjà.!\\
Le champ défini existe déjà dans cette classe.
La suite du message indique le nom de ce champ.\\

\lstinline!Ligne X : Type non autorisé.!\\
Le programme tente d'effectuer une opération sur une variable d'un
type sur lequel on ne peut pas l'appliquer (comparaison d'infériorité sur un booléen ...).
La suite du message indiquera s'il s'agit de l'opérande de droite ou de gauche.\\

\lstinline!Ligne X : Opération impossible.!\\
Le programme Deca cherche à effectuer une opération (comparaison...)
entre deux variables de types incompatibles pour cette opération.
La suite du message indique de quels types il
s'agit.\\

\lstinline!Ligne X : Type inexistant.!\\
Le type auquel le programme fait référence n'existe pas. La suite du message indique le nom
du type concerné.\\

\lstinline!Ligne X : Conversion impossible entre ces deux types.!\\
Le programme tente d'effectuer une conversion d'une expression ne correspondant pas aux conversions autorisées par le langage.\\

\lstinline!Ligne X : Types incompatibles pour l'affectation.!\\
Le programme cherche à effectuer une affectation entre deux variables ou objets de types incompatibles.
 Les types des deux variables sont précisés dans la suite.\\

\lstinline!Ligne X : Impossible de créer cette variable.!\\
La variable déclarée existe déjà, son nom est précisé dans la suite du message.\\

\lstinline!Ligne X : Cette variable n'est pas définie.!\\
La variable utilisée n'existe pas, son nom est précisé
dans la suite du message.\\

\lstinline!Ligne X : L'identifiant devrait être une classe.!\\
Un objet est instancié à l'aide du mot réservé new, mais le constructeur utilisé n'est pas celui d'une classe existante.
 L'identifiant concerné est précisé dans le reste du message.\\

\lstinline!Ligne X : Cette méthode n'existe pas.!\\
La méthode appelée n'existe pas pour la classe de l'objet sur
lequel elle est appelée. La suite du message indique le nom de la méthode concernée.\\

\lstinline!Ligne X : Un champ protégé ! \lstinline!n'est pas atteignable hors!
\lstinline!de la classe où! \lstinline! il est défini.!\\
Le champ utilisé est protégé et l'accès à ce champ ne se fait pas dans un cadre autorisé,
c'est-à-dire à l'intérieur de la classe ou dans une classe en héritant.
 Le nom de ce champ apparaît dans la suite du message.\\

\lstinline!Ligne X : Ce champ n'existe pas.!\\
Le champ utilisé n'existe pas pour la classe de l'objet sur
lequel il est appelé. La suite du message indique le nom du champ concerné.\\

\lstinline!Ligne X : Le mot clé! \lstinline! 'this' ne doit pas être appelé!
\lstinline!dans le bloc principal.!\\
L'expression \verb?this? est présente dans le bloc principal du programme. \verb?this? fait référence à la classe
dans laquelle il se trouve et doit donc être utilisé dans le corps d'une classe.\\

\lstinline!Ligne X : Le mot clé!\lstinline! 'return' ne doit pas être appelé!
\lstinline!dans le bloc principal.!\\
Le mot réservé \verb?return? est présent dans le bloc principal du programme. \verb?return? ne doit être utilisé qu'à
l'intérieur d'une méthode supposée retourner n'importe quel autre type que void, et en particulier
jamais dans le bloc principal.\\

\lstinline!Ligne X : Paramètres incorrects.!\\
Lors de l'appel à une méthode, il n'y a pas le bon nombre de paramètres.\\

\subsubsection{Autres Erreurs}
\lstinline!Ligne X : Warning.!\\
Si les avertissements sont demandés, signale que le programme effectue une action
susceptible de provoquer un comportement différent de celui voulu à l'exécution, mais
qui ne provoque pas d'erreur de compilation.\\

\lstinline!Ligne X : Erreur inconnue.!\\
Ceci indique que le compilateur a rencontré une erreur interne.\\

\subsection{Erreurs à l'exécution}
Cette catégorie rassemble les erreurs susceptibles d'être levées à l'exécution du programme assembleur généré.\\
Attention, certaines erreurs de programmation, en particulier l'accès à une variable non
initialisée, peuvent provoquer un comportement indéfini.\\

\lstinline!Erreur : pile_pleine!\\
La pile est remplie au maximum, ceci peut arriver par exemple si trop d'opérations sont demandées avant de pouvoir évaluer un résultat.\\

\lstinline!Erreur : tas_plein!\\
Le tas est plein, ceci peut arriver si trop d'objets sont instanciés ou s'ils sont trop volumineux.\\

\lstinline!Erreur : asin_argument_incorrect!\\
L'argument passé en paramètre de la fonction asin ne se trouve pas entre -1 et 1.\\

\lstinline!Erreur : division_par_zero!\\
Le programme tente de diviser par zéro ou par un nombre suffisamment petit pour être approximé à zéro.\\

\lstinline!Erreur : debordement_flottant!\\
Un des flottants manipulés par le programme n'est pas représentable par la norme IEEE754 en 32 bits (flottants en simple précision).\\

\lstinline!Erreur : pas_de_return_rencontre!\\
Une des méthodes supposée retourner autre chose qu'un type \verb?void? ne comporte pas d'instruction \verb?return?.\\

\lstinline!Erreur : erreur_conversion_classe!\\
Tentative de conversion d'un objet en un sous-type de son type actuel, ceci est correct contextuellement mais impossible.\\

\lstinline!Erreur : debordement_conversion_entier!\\
Un des entiers manipulés par le programme dépasse les bornes des entiers acceptés.\\

\section{Exemples d'utilisation}

Voici un fichier source cond0.deca et le fichier cond0.ass généré avec
\lstinline!decac -n cond0.deca! (on a volontairement omis de recopier le code
des classes Object et Math pour la lisibilité)
\begin{verbatim}
cond0.deca :
// Description:
//    tests conditionnelle
//
// Resultats:
//    ok
// 
// Cree le 01/01/2013

{
   if (1 >= 0) {
      println("ok"); 
   } else { 
      println("erreur");
   }
}

cond0.ass
Programme_principal :
     ADDSP #8                          ; Réservation de l'espace mémoire pour la table des méthodes et les variables globales
; Initialisation de la table des méthodes de Object
     LOAD #null, R0
     STORE R0, 1 (GB)
     LOAD code.Object.equals, R0
     STORE R0, 2 (GB)
; Initialisation de la table des méthodes de Math
     LEA 1 (GB), R0
     STORE R0, 3 (GB)
     LOAD code.Object.equals, R0
     STORE R0, 4 (GB)
     LOAD code.Math.cos, R0
     STORE R0, 5 (GB)
     LOAD code.Math.sin, R0
     STORE R0, 6 (GB)
     LOAD code.Math.asin, R0
     STORE R0, 7 (GB)
     LOAD code.Math.atan, R0
     STORE R0, 8 (GB)
; Partie principale du code assembleur.
; Déclaration des variables globales.
; Bloc d'instructions principal
; Ligne  10 : if
; Ligne  10 : début de >= 
     LOAD #1, R2
     LOAD #0, R1
     CMP R1, R2
     BLT Sinon.2
; Ligne  10 : fin de comparaison 
     WSTR "ok"                         ; Ligne  11 : Print
     WNL
     BRA fin_if.1
Sinon.2 :
     WSTR "erreur"                     ; Ligne  13 : Print
     WNL
fin_if.1 :
     HALT
Erreurs :
asin_argument_incorrect :
     WSTR "Erreur : asin_argument_incorrect"
     WNL
     ERROR
pas_de_return_rencontre :
     WSTR "Erreur : pas_de_return_rencontre"
     WNL
     ERROR
Classes :
; ----------------------------------------------------------
; ----------- Classe Object
; ----------------------------------------------------------

; Code de la classe Object

; ----------------------------------------------------------
; ----------- Classe Math
; ----------------------------------------------------------

; Code de la classe Math

\end{verbatim}

\end{document}
