\documentclass[12pt]{article}

\usepackage[french]{babel}
\usepackage[T1]{fontenc}
\usepackage[utf8]{inputenc}
\usepackage{fancyhdr}
\usepackage{lastpage}
\usepackage{graphicx}
\usepackage{verbatim}
\usepackage{amsmath}

\usepackage{listings}
\lstset{
  sensitive=f,
  morecomment=[l]--,
  morestring=[d]",
  showstringspaces=false,
  basicstyle=\small\ttfamily,
  keywordstyle=\bf\small,
  commentstyle=\itshape,
  stringstyle=\sf,
  extendedchars=true,
  columns=[c]fixed
}


% CI-DESSOUS: conversion des caractères accentués UTF-8 
% en caractères TeX dans les listings...
\lstset{
  literate=%
  {À}{{\`A}}1 {Â}{{\^A}}1 {Ç}{{\c{C}}}1%
  {à}{{\`a}}1 {â}{{\^a}}1 {ç}{{\c{c}}}1%
  {É}{{\'E}}1 {È}{{\`E}}1 {Ê}{{\^E}}1 {Ë}{{\"E}}1% 
  {é}{{\'e}}1 {è}{{\`e}}1 {ê}{{\^e}}1 {ë}{{\"e}}1%
  {Ï}{{\"I}}1 {Î}{{\^I}}1 {Ô}{{\^O}}1%
  {ï}{{\"i}}1 {î}{{\^i}}1 {ô}{{\^o}}1%
  {Ù}{{\`U}}1 {Û}{{\^U}}1 {Ü}{{\"U}}1%
  {ù}{{\`u}}1 {û}{{\^u}}1 {ü}{{\"u}}1%
}

%%%%%%%%%%
% TAILLE DES PAGES (A4 serré)

\setlength{\parindent}{0cm}
\setlength{\parskip}{1ex}
\setlength{\textwidth}{17cm}
\setlength{\textheight}{24cm}
\setlength{\oddsidemargin}{-.7cm}
\setlength{\evensidemargin}{-.7cm}
\setlength{\topmargin}{-.5in}

%%%%%%%%%%
% EN-TÊTES ET PIED DE PAGES

\pagestyle{fancyplain}
\renewcommand{\headrulewidth}{0pt}
\addtolength{\headheight}{1.6pt}
\addtolength{\headheight}{2.6pt}
\lfoot{}
\cfoot{}
\rfoot{\footnotesize\sf \thepage/\pageref{LastPage}} % numero de page
\lhead{\footnotesize\sf GeneTeX} % en-tete Gauche

%%%%%%%%%%
% TITRE DU DOCUMENT

\title{\textbf{GeneTeX \\
  User Manual}}

\author{Mehdi \textsc{Bouksara} - Théo \textsc{Merle} - Marceau \textsc{Thalgott} \\}

\date{May - June 2013}


%%%%%%%

\begin{document}

\maketitle
~\\
\begin{center}
\includegraphics[width=10cm]{genetex.eps}
\end{center}

\newpage
\section{Introduction}

GeneTeX is an image-to-LaTeX translator. It is able to read an image containing written text, and to generate a LaTeX document containing the text written in the original image, and to give information about the liability of the retranscription.\\
GeneTeX comes with ABSOLUTELY NO WARRANTY, to the extent permitted by applicable law. To be more accurate, GeneTeX is not able to generate a perfectly accurate retranscription of a given document, as the quality of the character recognition depends on various factors such as the resolution of the original image, or the quality of the writing.\\
\\
When used without any options, and when all the parameters are correct, GeneTeX generates a .tex document, with the same name than the input image, and doesn't write anything on standard output.

% \subsection{Package \lstinline!Code!}

\subsection{Synopsis}
\lstinline{$genetex [source [-v] [-r destination]] | [-h] }\\ %$

\subsection{Options}
\begin{itemize}
       \item[] \lstinline!-h! (help)\\
         Writes the help text of GeneTeX on standard output. Must be used alone.
       \item[] \lstinline!-v! (verbose)\\
         Allows GeneTeX to print verbose messages giving indications on the transcription progress.
       \item[]\lstinline!-r! (rename)\\
         Specifies destination as the name of the generated LaTeX document.
\end{itemize}


\section{Recognized fonts and writing styles}

GeneTeX is only able to recognize the original LaTeX font (Computer Modern).\\
It won't be able to recognise other fonts, and won't recognise handwritten text either.\\
The characters in the image must not overlap horizontally or vertically.\\
The image must not contain any element that is not text to be recognized.

\section{Known bugs and limitations}

\begin{itemize}
  \item[] - GeneTeX can only recognize images with no scanning defects or noise.
  \item[] - GeneTeX cannot recognize characters if it is not perfectly horizontal.
  \item[] - GeneTeX may be wrong when recognizing characters such as 'l', 'o', '.', which can be mistaken for a look-alike character. Reading the LaTeX output will probably be necessary.
\end{itemize}

\end{document}
